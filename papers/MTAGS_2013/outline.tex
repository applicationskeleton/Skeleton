\documentclass[10pt,a4paper]{article}
\usepackage[latin1]{inputenc}
\usepackage{amsmath}
\usepackage{amsfonts}
\usepackage{amssymb}
\usepackage[pdftex]{graphicx}
  \DeclareGraphicsExtensions{.pdf,.jpeg,.png}
\usepackage{subfigure}
\title{Skeleton Paper Outline}
\begin{document}
\maketitle
\section{Introduction}
In this section, we state the motivation, challenge and contribution of the Skeleton.
\begin{itemize}
\item {Motivation}: Computer scientists who work on tools and systems meant to
support or enable a variety of distributed computing applications want to use real
applications to prove that the systems they design actually help those applications.
However, in many cases, compiling and executing real applications can be time
consuming. Some of applications (or data) are privately accessible; some use legacy
code and are dependent on out-of-date libraries; and some are hard to understand
based on the computer scientists knowledge of the background, when it might not
be necessary for the computer scientists to understand the science problem.
These issue led us to the idea of an Application Skeleton - to provide computer
scientists a powerful and easy-to-use tool to build synthetic applications that represent
real applications, with runtime, I/O, and intertask communication close to identical to those of
the real applications. This allows computer scientists to focus on the system they are
building; they can work with the simpler Skeleton applications and be sure that their
work will also be applicable to the real applications.

\item {Challenge}: The challenge is to provide an easy-to-use programming
(specification) model to express an application with an acceptable performance
difference between the Skeleton application and the real application.

\item {Contribution}: 1. An application abstraction that gives users good expressiveness
to capture the key performance elements of applications. 2. A versatile Skeleton
implementation that is interoperable with mainstream workflow frameworks and
systems (e.g., Shell, Pegasus and Swift).

\end{itemize}

\section{Application Examples}
This section lists a couple of common application examples: a bag-of-tasks, map-reduce applications, and multi-stage applications (DOCK, PageRank, Montage).

\section{Skeleton Interface Design}
Skeleton abstracts an application as a number of stages, and each stage is abstracted as a number of tasks. 

\section{Skeleton App Performance}
I will use Montage, BLAST, CyberShake as sample applications to compare the skeleton app performance and the real application performance. The platform I can think of right now is BG/P, and I can run the scripts through AMFS. Ideally, skeleton and real app should deliver identical performance.
\end{document}