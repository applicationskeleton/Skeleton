%%%%%%%%%%%%%%%%%%%%%%%%%%%%%%%%%%%%%%%%%
% Plain Cover Letter
% LaTeX Template
% Version 1.0 (28/5/13)
%
% This template has been downloaded from:
% http://www.LaTeXTemplates.com
%
% Original author:
% Rensselaer Polytechnic Institute 
% http://www.rpi.edu/dept/arc/training/latex/resumes/
%
% License:
% CC BY-NC-SA 3.0 (http://creativecommons.org/licenses/by-nc-sa/3.0/)
%
%%%%%%%%%%%%%%%%%%%%%%%%%%%%%%%%%%%%%%%%%

%----------------------------------------------------------------------------------------
%	PACKAGES AND OTHER DOCUMENT CONFIGURATIONS
%----------------------------------------------------------------------------------------

\documentclass[11pt]{letter} % Default font size of the document, change to 10pt to fit more text

\usepackage{newcent} % Default font is the New Century Schoolbook PostScript font 
%\usepackage{helvet} % Uncomment this (while commenting the above line) to use the Helvetica font

% Margins
\topmargin=-1in % Moves the top of the document 1 inch above the default
\textheight=8.5in % Total height of the text on the page before text goes on to the next page, this can be increased in a longer letter
\oddsidemargin=-10pt % Position of the left margin, can be negative or positive if you want more or less room
\textwidth=6.5in % Total width of the text, increase this if the left margin was decreased and vice-versa

\let\raggedleft\raggedright % Pushes the date (at the top) to the left, comment this line to have the date on the right


\begin{document}

Changes from eScience conference paper:

The main change in this paper are in the discussion of manual vs automated skeleton parameter estimation.  The automated skeleton parameter work is completely new.

In general, all sections of the paper have been changed to reflect this, but mostly with minor changes.

The major changes are:

\begin{itemize}
\item Section 3, Determining Skeleton Parameters, is almost all new, though some of Section 3.1 was previously scattered elsewhere in the paper, mostly in Section 4.
\item Section 4, Performance Evaluation, has had the main points condensed from the previous paper, and also has had information added based on the automated skeleton parameter methods discussed in Section 3.
\item Section 7, Conclusions and Future Work, has had some discussion added about how Skeletons can be used in education.
\end{itemize}


\end{document}